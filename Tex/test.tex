% Preamble stuff
\documentclass{article} % This can be used to specify the template you will use for your document
\setlength{\parskip}{1em} % Sets the spacing between paragraphs

\usepackage[utf8]{inputenc} % Input encoding set to UTF-8 (allows emojis and stuff) (careful what compiler you use though)
\usepackage[margin=1.5in]{geometry} % Use this to specify margins
\usepackage{amsmath,amssymb} % American Math Society, allows for symbols, improves existing macros and adds for environments such as align
\usepackage{graphicx} % Allows us to embed images
\usepackage{indentfirst} % Indents the first paragraph

\title{Test}
\author{John Doe}
\date{August 2021}

% Document stuff
\begin{document} % Indicates the beginning of a document

\maketitle{Yeah}

\section{Introduction}
This document will be used to demonstrate various base functions available in latex. 

\section{Text Formatting}
\textit{This is italicized text}
\textbf{This is bolded text}
\underline{This is underlined text}

\section{Math Formatting}
Display Style Math Example:
\[f(x)=(x+2)^2-9\] % Functions must be encased between \[\]

Align * environment variation: % Must begin and end an align* environment, and use ampersands where you want equations to be aligned. (note that removing the star makes the equations numbered)
\begin{align*}
f(x)&=(x+2)^2-9\\
f(1)&=(1+2)^2-9\\ 
&=9-9\\
&=0
\end{align*}
\indent Numbered align* with multiple equations per line:
\begin{align}
x+2y&=8 & x-y&=-1\\
x+2y&=8 & 2x-2y&=-2
\end{align}
\begin{align*}
x+2y&=8\\
2x-2y&=-2\\
3x&=6\\
x&=2\\
\end{align*}

Inline equations:

Example function: \(f(x)=x^2+2x+2\) has imaginary roots. % Generally inline equations take up less vertical space

This is a function of \(x\). % Use \(\) to denote an equation; may be used to italicize a variable letter

\end{document} % Indicates the end of a document
